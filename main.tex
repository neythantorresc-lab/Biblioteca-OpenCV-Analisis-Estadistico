\documentclass[12pt]{article}

\usepackage[spanish]{babel}
\usepackage[utf8]{inputenc}
\usepackage{graphicx}
\usepackage{amsmath}
\usepackage{listings}
\usepackage{xcolor}
\usepackage{geometry}

\geometry{margin=2.5cm}

\title{Análisis Estadístico en Tiempo Real usando OpenCV}
\author{Tu Nombre}
\date{\today}

\lstset{
    language=Python,
    basicstyle=\ttfamily\small,
    keywordstyle=\color{blue},
    stringstyle=\color{red},
    commentstyle=\color{green},
    numbers=left,
    numberstyle=\tiny,
    breaklines=true
}

\begin{document}

\maketitle

\section{Introducción}

La visión por computadora es una rama de la inteligencia artificial que permite a las máquinas interpretar información visual del entorno. Una de las bibliotecas más utilizadas en este campo es OpenCV (Open Source Computer Vision Library), la cual permite el procesamiento de imágenes y video en tiempo real.

En este documento se presenta una aplicación desarrollada en Python que realiza un análisis estadístico de las intensidades de una imagen capturada desde la cámara en tiempo real.

\section{Instalación}

Para utilizar OpenCV en Python es necesario instalar la biblioteca mediante el siguiente comando:

\begin{verbatim}
pip install opencv-python numpy matplotlib
\end{verbatim}

Las bibliotecas utilizadas son:

\begin{itemize}
\item OpenCV: captura y procesamiento de imagen.
\item NumPy: cálculos estadísticos.
\item Matplotlib: visualización del histograma.
\end{itemize}

\section{Fundamento Teórico}

En una imagen en escala de grises, cada píxel puede tomar valores entre 0 y 255. Esta intensidad puede modelarse como una variable aleatoria discreta:

\[
X \in \{0,1,2,...,255\}
\]

\subsection{Esperanza Matemática}

La media se define como:

\[
E(X) = \sum_{i=0}^{255} x_i P(x_i)
\]

\subsection{Varianza}

La varianza mide la dispersión:

\[
Var(X) = E(X^2) - [E(X)]^2
\]

\subsection{Desviación Estándar}

\[
\sigma = \sqrt{Var(X)}
\]

\section{Descripción de la Aplicación}

La aplicación realiza los siguientes pasos:

\begin{enumerate}
\item Captura video desde la cámara.
\item Convierte la imagen a escala de grises.
\item Convierte la matriz de píxeles en un vector estadístico.
\item Calcula media, varianza y desviación estándar.
\item Calcula la distribución empírica de probabilidad.
\item Estima la probabilidad de que la intensidad sea mayor a 200.
\item Muestra los resultados en tiempo real.
\end{enumerate}

\section{Código Fuente}

\begin{lstlisting}
# AQUÍ DEBES PEGAR TU CÓDIGO COMPLETO DE PYTHON
\end{lstlisting}

\section{Resultados}

Durante las pruebas realizadas se observó que:

\begin{itemize}
\item La media aumenta cuando la iluminación es mayor.
\item La varianza incrementa cuando existe mayor contraste.
\item La probabilidad P(X > 200) aumenta ante objetos blancos.
\end{itemize}

\section{Conclusión}

Se demostró que es posible aplicar conceptos de probabilidad y estadística al análisis de imágenes en tiempo real. La intensidad de píxel puede modelarse como una variable aleatoria discreta, permitiendo calcular métricas como esperanza matemática y varianza.

El uso de OpenCV facilita la captura y procesamiento eficiente de datos visuales, integrando teoría matemática con aplicaciones prácticas en ingeniería de sistemas.

\end{document}
